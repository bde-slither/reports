\documentclass[conference]{IEEEtran}
\IEEEoverridecommandlockouts
% The preceding line is only needed to identify funding in the first footnote. If that is unneeded, please comment it out.
\usepackage{cite}
\usepackage{amsmath,amssymb,amsfonts}
\usepackage{algorithmic}
\usepackage{graphicx}
\usepackage{textcomp}
\def\BibTeX{{\rm B\kern-.05em{\sc i\kern-.025em b}\kern-.08em
    T\kern-.1667em\lower.7ex\hbox{E}\kern-.125emX}}
\begin{document}

\title{Paper Title*\\
{\footnotesize \textsuperscript{*}Note: Sub-titles are not captured in Xplore and
should not be used}
\thanks{Identify applicable funding agency here. If none, delete this.}
}

\author{\IEEEauthorblockN{1\textsuperscript{st} Given Name Surname}
\IEEEauthorblockA{\textit{dept. name of organization (of Aff.)} \\
\textit{name of organization (of Aff.)}\\
City, Country \\
email address}
\and
\IEEEauthorblockN{2\textsuperscript{nd} Given Name Surname}
\IEEEauthorblockA{\textit{dept. name of organization (of Aff.)} \\
\textit{name of organization (of Aff.)}\\
City, Country \\
email address}
\and
\IEEEauthorblockN{3\textsuperscript{rd} Given Name Surname}
\IEEEauthorblockA{\textit{dept. name of organization (of Aff.)} \\
\textit{name of organization (of Aff.)}\\
City, Country \\
email address}
\and
\IEEEauthorblockN{4\textsuperscript{th} Given Name Surname}
\IEEEauthorblockA{\textit{dept. name of organization (of Aff.)} \\
\textit{name of organization (of Aff.)}\\
City, Country \\
email address}
\and
\IEEEauthorblockN{5\textsuperscript{th} Given Name Surname}
\IEEEauthorblockA{\textit{dept. name of organization (of Aff.)} \\
\textit{name of organization (of Aff.)}\\
City, Country \\
email address}
\and
\IEEEauthorblockN{6\textsuperscript{th} Given Name Surname}
\IEEEauthorblockA{\textit{dept. name of organization (of Aff.)} \\
\textit{name of organization (of Aff.)}\\
City, Country \\
email address}
}

\maketitle

\section{Proposed Work}
\subsection*{The Game}
In this project we aim to create a, AI based, multi-player snake game inspired by the online game \textit{slither.io}. Food items would appear randomly on the game board. The snakes would gain a unit length after every second food item consumed on the board. A snake would be considered dead if it hits the boundary or bumps into other snakes. This will make the snake move away from other snakes and survive longer, at the same time allowing bigger snakes to trap smaller snakes and kill them. The game would spawn two or more snakes and play until only one snake remains.\newline\par

\subsection{Tools}
We plan to use the Gym toolkit developed by OpenAI\cite{n1}. It was built to enable a user to train his own programmed artificially intelligent agents remotely. The user interacts with the server with two simple function calls: make and step. These take care of starting the game server and taking input for the agents to act in the environment. It returns a tuple consisting of the current game pixels, the reward received since the last step, a boolean to check if the bot has died, and latency information
\newline\par
We will first build the game environment using a python graphics toolkit and attach it with \textit{gym}. We will then implement our learning algorithm using TensorFlow\cite{n2} and attach it to the environment through Gym as the agent. \newline\par
\subsection{Performance Metrics}
In order to rate the performance of each snake, the following equation is proposed to calculate the score
\begin{equation}
Score=\lambda_1 Points+ \lambda_2 Kills+\lambda_3Length
\end{equation}
The value of \({\lambda_1,\lambda_2}\) and \({\lambda_3}\) can be experimented with to find a optimal value.

\subsection{Algorithms}
We will first try to implement a simple neural network for learning the Q-function and use its performance values as baseline to compare further algorithms.  \newline\par
   
   
\subsection{Additional Objective}
\begin{itemize}
	\item{Explore rotational invariance: The training algorithm will see only a small square section of the environment. Since the grids are square, the strategy should be invariant to 90 (degree) rotation. Therefore it can be considered that the snake is always moving above. This can potentially reduce the search space by a factor of 4 and reduce learning time.}
	\item{Prevent self-play instability: It may happen that the snakes, in order to survival, choose a particular region of the board and stay local to it(eat food and avoid boundary). The game may continue indefinitely without a decisive result. }
\end{itemize}
\section*{}

\begin{thebibliography}{00}
\bibitem{n1}{1606.01540,Greg Brockman, Vicki Cheung, Ludwig Pettersson, Jonas Schneider, John Schulman, Jie Tang and Wojciech Zaremba,OpenAI Gym,2016}
\bibitem{n2}https://www.tensorflow.org
\end{thebibliography}

\end{document}
